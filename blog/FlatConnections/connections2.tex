\documentclass[]{article}
\usepackage{lmodern}
\usepackage{amssymb,amsmath}
\usepackage{ifxetex,ifluatex}
\usepackage{fixltx2e} % provides \textsubscript
\ifnum 0\ifxetex 1\fi\ifluatex 1\fi=0 % if pdftex
  \usepackage[T1]{fontenc}
  \usepackage[utf8]{inputenc}
\else % if luatex or xelatex
  \ifxetex
    \usepackage{mathspec}
  \else
    \usepackage{fontspec}
  \fi
  \defaultfontfeatures{Ligatures=TeX,Scale=MatchLowercase}
  \newcommand{\euro}{€}
\fi
% use upquote if available, for straight quotes in verbatim environments
\IfFileExists{upquote.sty}{\usepackage{upquote}}{}
% use microtype if available
\IfFileExists{microtype.sty}{%
\usepackage{microtype}
\UseMicrotypeSet[protrusion]{basicmath} % disable protrusion for tt fonts
}{}
\usepackage[margin=1in]{geometry}
\usepackage{hyperref}
\PassOptionsToPackage{usenames,dvipsnames}{color} % color is loaded by hyperref
\hypersetup{unicode=true,
            pdftitle={Flat connections and the fundamental group},
            pdfauthor={Guillermo Gallego},
            pdfborder={0 0 0},
            breaklinks=true}
\urlstyle{same}  % don't use monospace font for urls
\usepackage{graphicx,grffile}
\makeatletter
\def\maxwidth{\ifdim\Gin@nat@width>\linewidth\linewidth\else\Gin@nat@width\fi}
\def\maxheight{\ifdim\Gin@nat@height>\textheight\textheight\else\Gin@nat@height\fi}
\makeatother
% Scale images if necessary, so that they will not overflow the page
% margins by default, and it is still possible to overwrite the defaults
% using explicit options in \includegraphics[width, height, ...]{}
\setkeys{Gin}{width=\maxwidth,height=\maxheight,keepaspectratio}
\setlength{\parindent}{0pt}
\setlength{\parskip}{6pt plus 2pt minus 1pt}
\setlength{\emergencystretch}{3em}  % prevent overfull lines
\providecommand{\tightlist}{%
  \setlength{\itemsep}{0pt}\setlength{\parskip}{0pt}}
\setcounter{secnumdepth}{0}

%%% Use protect on footnotes to avoid problems with footnotes in titles
\let\rmarkdownfootnote\footnote%
\def\footnote{\protect\rmarkdownfootnote}

%%% Change title format to be more compact
\usepackage{titling}

% Create subtitle command for use in maketitle
\providecommand{\subtitle}[1]{
  \posttitle{
    \begin{center}\large#1\end{center}
    }
}

\setlength{\droptitle}{-2em}

  \title{Flat connections and the fundamental group}
    \pretitle{\vspace{\droptitle}\centering\huge}
  \posttitle{\par}
    \author{Guillermo Gallego}
    \preauthor{\centering\large\emph}
  \postauthor{\par}
      \predate{\centering\large\emph}
  \postdate{\par}
    \date{July 1, 2020}


% Redefines (sub)paragraphs to behave more like sections
\ifx\paragraph\undefined\else
\let\oldparagraph\paragraph
\renewcommand{\paragraph}[1]{\oldparagraph{#1}\mbox{}}
\fi
\ifx\subparagraph\undefined\else
\let\oldsubparagraph\subparagraph
\renewcommand{\subparagraph}[1]{\oldsubparagraph{#1}\mbox{}}
\fi

\usepackage{amscd, amsmath}

\begin{document}
\maketitle

In this post we are going to study the correspondence between flat
connections and representations of the fundamental group, sometimes
called the ``Riemann-Hilbert correspondence''. This is a very rich
result in that it draws together several fields of Mathematics and leads
to new interesting problems like the study of certain moduli spaces or
the relationship to Yang-Mills equation. Moreover, this correspondence
is the first step to fully understand another correspondence known as
the Nonabelian Hodge Theorem.

There are several ways of proving this correspondence. One is through
holonomy, and it is maybe the more straightforward since it directly
relates flat connections with representations of the fundamental group.
In this post however, given that in my last post {[}CITAR{]} we already
gave a correspondence between representations of the fundamental group
and covering spaces, we are going to prove the correspondence by
relating flat connections to covering spaces (also called \emph{local
systems} in this context).

With this point of view, a general ``global'' proof of the
correspondence can be given by using the Frobenius theorem. We are going
to sketch this proof, but I prefer to study in detail the proof for
matrix groups, which is more simple and straightforward, and maybe fits
better with the ``cocycle approach'' that I have been mantaining in my
previous posts.

\section{Principal bundles and local
systems}\label{principal-bundles-and-local-systems}

During all this post we will be using the notions from my post on
Torsors and Cocycles {[}CITAR{]}, and also we will make use of the
correspondence between covering spaces and representations that I gave
in my previous post {[}CITAR{]}.

Let me recall that for \(X\) a topological space and \(\mathcal{G}\) a
sheaf of groups over \(X\), a \(\mathcal{G}\) torsor is a sheaf of sets
on \(X\) with nonempty stalks which is endowed with a free and
transitive action of \(\mathcal{G}\). What we saw in my post about
Torsors and cocycles {[}CITAR{]} is that the functor that sends every
\(\mathcal{G}\)-torsor to its set of transition functions gives an
equivalence of categories between the category of \(\mathcal{G}\) torsor
and the action groupoid \([Z^1(X,\mathcal{G}),C^0(X,\mathcal{G})]\),
where these two sets denote the sets of equivalence classes by
refinement of pairs \((\mathfrak{U},f)\), with \(\mathfrak{U}\) an open
cover of \(X\) and \(f\) a Čech \(1\)-cocycle (a Čech \(0\)-cochain,
respectively).

For our purposes, we will fix now once and for all \(X\) a smooth
manifold and \(G\) a Lie group, and consider the sheaves \(G\), which
maps any open set \(U\subset X\) to the group \(C^\infty(U,G)\) of
\(G\)-valued smooth functions, and \(\underline{G}\), which maps any
open set \(U\subset X\) to the group of locally constant functions
\(U\rightarrow G\).

We will call a \(G\)-torsor on \(X\) a \emph{principal \(G\)-bundle}
over \(X\), and we will call a \(\underline{G}\)-torsor on \(X\) a
\emph{\(G\)-local system} over \(X\).

Recall that in my previous post {[}CITAR{]} we called elements of
\(Z^1(\mathfrak{U},\underline{G})\) by the name of
\(\mathfrak{U}\)-based \(G\)-covering spaces. Therefore, a \(G\)-local
system is an equivalence class by refinement of \(G\)-coverings. In that
post we also proved the \emph{monodromy theorem} which said that the
groupoid
\([Z^1(\mathfrak{U},\underline{G}),C^0(\mathfrak{U},\underline{G})]\)
was equivalent to the groupoid \([\mathrm{Hom}(\pi_1(X,x_0),G),G]\) of
\(G\)-representations of the fundamental groupo with the conjugation
action, which we called the \emph{Betti groupoid}, \(\mathfrak{U}\)
being an open cover satisfying certain ``good'' topological conditions.
Since we are on a smooth manifold, every open cover can be refined to
one that is ``good'' in that sense, and thus we get an equivalence
between the category of \(G\)-local systems and the Betti groupoid.

The purpose of this post is to prove that the category of \(G\)-local
systems is equivalent to another ``differential geometric'' category,
the category of \emph{flat bundles}. As a consequence, this equivalence
will yield an equivalence between this category of flat bundles and the
Betti groupoid.

\section{Vector bundles}\label{vector-bundles}

We will start by proving this equivalence for vector bundles, so we
better define what these are.

Consider the \emph{sheaf of smooth complex-valued functions}
\(C^\infty_X\) on \(X\) sending every open set \(U\subset X\) to the set
\(C^\infty(X,\mathbb{C})\). A \emph{sheaf of \(C^\infty_X\) modules} is
a sheaf of abelian groups \(E\) on \(X\) such that, on any
\(U\subset X\), the group \(E(U)\) is a \(C^\infty_X(U)\)-module and
such that the restriction homomorphisms \(E(V) \rightarrow E(U)\) are
\(C^\infty_X(U)\)-linear.

A \emph{complex vector bundle of rank \(n\) on \(X\)} is a locally free
sheaf \(E\) of \(C^\infty_X\) modules. This means that there exists an
open cover \(\mathfrak{U}\) of \(X\) and isomorphisms
\(\varphi_U: E|_U \rightarrow C^\infty_X|_U ^n\) for every \(U \in X\).
The pair \((\mathfrak{U},\varphi)\), with
\(\varphi=\{\varphi_U\}_{U\in \mathfrak{U}}\) is called a
\emph{trivialization} of \(E\).

To any rank \(n\) complex vector bundle \(E\), fixing a trivialization
\((\mathfrak{U},\varphi)\), we can associate its set of transition
functions \(g_{UV}:U \cap V \rightarrow \mathrm{GL}(n,\mathbb{C})\),
defined as

\[
g_{UV}=\varphi_V|_{U\cap V} \circ \varphi_U^{-1}|_{U\cap V},
\]

for \(U,V \in \mathfrak{U}\), \(U\cap V \neq \varnothing\). Clearly the
\(g_{UV}\) define a cocycle, so we get a map

\[
\{ \text{Rank $n$ vector bundles} \} \longrightarrow Z^1(\mathfrak{U},\mathrm{GL}(n,\mathbb{C})).
\]

A \emph{gauge transformation} of a vector bundle \(E\) is a
\(C^\infty_X\)-linear sheaf automorphism \(\xi:E\rightarrow E\). Fixing
a trivialization \((\mathfrak{U},\varphi)\), to any gauge transformation
\(\xi\) we can associate the element
\(f\in C^0(\mathfrak{U},\mathrm{GL}(n,\mathbb{C}))\) defined as

\[
f_U=\varphi_U \circ \xi|_U \circ \varphi_U^{-1},
\]

for \(U\in \mathfrak{U}\).

Thus, if we denote by \(\mathbf{Vect}_n\) the category whose objects are
vector bundles on \(X\) of rank \(n\) and where morphisms are given by
gauge transformations, by making a choice of trivialization on any
vector bundle, we have defined a functor

\[
\mathbf{Vect}_n \longrightarrow [Z^1(X,\mathrm{GL}(n,\mathbb{C})), C^0(X,\mathrm{GL}(n,\mathbb{C}))].
\]

This functor is fully faithful. Fix \(E\) is a vector bundle and we fix
\((\mathfrak{U},\varphi)\) a trivialization of \(E\). Clearly, the map
\(\xi\mapsto f\), with \(f\) defined as above gives a bijection between
gauge transformations and elements of
\(C^0(\mathfrak{U},\mathrm{GL}(n,\mathbb{C}))\).

This functor is essentially surjective. Let \((\mathfrak{U},g)\), with
\(g \in Z^1(\mathfrak{U},\mathrm{GL}(n,\mathbb{C}))\) be a pair
representing an element of \(Z^1(X,\mathrm{GL}(n,\mathbb{C}))\). The way
to recover now the vector bundle is similar to how we recovered a torsor
from a cocycle. Define the presheaf

\[
E(U)= \coprod_{V \in \mathfrak{U}} C^\infty_X(U\cap V)^n / \sim,
\]

where \(f \in C^\infty_X(U\cap V)^n\) and
\(f' \in C^\infty_X(U\cap V')^n\), with \(V\cap V'\) are related by
\(\sim\) if

\[
f|_{U\cap V \cap V'} = g_{V V'} f'|_{U\cap V \cap V'}.
\]

This presheaf verifies the sheaf condition by construction and it is
clearly locally free of rank \(n\). If \((\mathfrak{V},\varphi)\) is any
trivialization of \(E\), then it is easy to check that, after passing to
a common refinement, its transition functions are on the same orbit as
the refinement of the cocycle \(g\) by a \(0\)-cochain.

Therefore, the category \(\mathbf{Vect}_n\) is equivalent to that of
\(\mathrm{GL}(n,\mathbb{C})\)-torsors (that is, principal
\(\mathrm{GL}(n,\mathbb{C})\)-bundles).

\section{Connections in vector
bundles}\label{connections-in-vector-bundles}

Let \(E\) be a vector bundle and consider the bundles \(\Omega^k_X\)
consisting on complex-valued smooth differential \(k\)-forms on \(X\).
For example, \(\Omega^1_X\) is the \emph{cotangent bundle} of \(X\)
(actually, it is the sheaf of sections of the cotangent bundle, but we
are already regarding bundles as locally free sheaves).

\textbf{Definition}. A \emph{connection} \(D\) on \(E\) is a
\(\mathbb{C}\)-linear operator

\[
D: E \rightarrow E \otimes \Omega^1_X
\]

such that

\[
D(fs) = sdf + f Ds,
\]

for \(f\in C^\infty_X(U)\) and \(s\in E(U)\), for every open subset
\(U\subset X\).

Let \(D\) be a connection on a vector bundle \(E\) and take an open set
\(U\in \mathfrak{U}\) in some trivialization \((\mathfrak{U},\varphi)\)
of \(E\). We define a \emph{frame of \(E\) in \(U\)} to be a basis
\(\{e_1,...,e_n\}\) of \(E(U)\), given that it is a free
\(C^\infty_X(U)\)-module. For any \(e_i\) of the frame, the connection
acts as

\[
D e_i = \sum_j e_j A^j_i,
\]

for \(A^j_i \in \Omega^1_X(U)\). Using matrix notation, regarding
\(e=(e_i)\) as a row vector and \(A=(A^j_i)\) as a square matrix, we get

\[
De= eA.
\]

Now, given any other section \(s\in E(U)\), we can write
\(s=\sum_i s^i e_i\), for \(s^i \in C^\infty_X(U)\) and we have

\[
D s = \sum_i (d s^i e_i + s^i D e_i) = \sum_i ds^i e_i + s^i e_j A^i_j  = (d+A) s.
\]

The matrix \(A\) is called the \emph{connection \(1\)-form} of \(D\) on
\(U\).

\textbf{Definition}. Let \(D\) be a connection on a vector bundle \(E\).
We define the \emph{curvature} of \(D\) as the operator

\[
D^2: E \rightarrow E \otimes \Omega^2_X.
\]

The curvature is a \(C^\infty_X\)-linear map since

\[
D^2(fs)=D(sdf + fDs)=Ds \wedge df + df \wedge Ds + f D^2 s= fD^2 s,
\]

for \(s\in E(U)\) and \(f \in C^\infty_X(U)\).

Locally, in some trivializing open set, we have

\[
D^2(e)=D(eA) = De \wedge A + edA = e(A\wedge A + dA)=e F_A,
\]

for \(F_A= dA + A\wedge A\) a matrix of \(2\)-forms which we call the
\emph{curvature \(2\)-form}.

Fix \(E\) a vector bundle and consider the group \(\mathcal{G}_E\) of
gauge transformations of \(E\) and the set \(\mathcal{A}_E\) of all
connections on \(E\). We have a natural action of \(\mathcal{G}_E\) on
\(\mathcal{A}_E\) by conjugation: if \(s\in E(U)\),

\[
(\xi \cdot D) (s) = \xi \circ D \circ \xi^{-1}.
\]

Now, take \(U\in \mathfrak{U}\) for \((\mathfrak{U},\varphi)\) some
trivialization and \(f_U= \varphi_U \circ \xi|_U \circ \varphi_U^{-1}\).
If we choose the local frame \(e\) as the inverse image through
\(\varphi_U\) of the canonical basis and we consider \(A\) the
connection 1-form of \(D\) in this frame we have

\[
e (\xi \cdot A) = (\xi \cdot D) (e) = \xi D(\xi^{-1}(e)) = \xi D(ef_U^{-1}) = \xi (De f_U^{-1} + e df_U^{-1}) = e(f_U A f_U^{-1}  + f_U df_U^{-1}),
\]

so

\[
\xi \cdot A =f_U A f_U^{-1}  + f_U df_U^{-1}.
\]

The curvature of the gauge-transformed connection now is

\[
(\xi \cdot D)^2= \xi \circ D \circ \xi^{-1} \circ \xi \circ D \circ \xi^{-1} = \xi \circ D^2 \circ \xi^{-1}.
\]

Thus, in the frame defined above,

\begin{align*}
e F_{\xi \cdot A} &= \xi(D^2(\xi^{-1}(e))) = D^2(e f_U^{-1}) f_U = D(D(e) f_U^{-1} + e df_U^{-1}) f_U \\
&= (D^2(e) f_U^{-1} - D(e) \wedge df_U^{-1} + D(e) \wedge df_U^{-1} ) f_U =  D^2(e) = e F_A.
\end{align*}

\section{Flat connections}\label{flat-connections}

\textbf{Definition}. We say that a connection \(D\) on a vector bundle
\(E\) is flat if its curvature vanishes, that is, \(D^2=0\). A
\emph{flat vector bundle} is a pair \((E,D)\), where \(E\) is a vector
bundle and \(D\) is a flat connection on \(D\).

If we denote by \(\mathcal{F}_E\) the set of flat connections on a
vector bundle \(E\), the formulas above show that the action of the
group of gauge transformations \(\mathcal{G}_E\) on \(\mathcal{A}_E\)
descends to an action on \(\mathcal{F}_E\). And in consequence the
functor relating vector bundles and principal
\(\mathrm{GL}(n,\mathbb{C})\)-bundles descends to a functor

\[
\mathbf{FlatVect}_n \longrightarrow [Z^1(X,\mathrm{GL}(n,\mathbb{C})), C^0(X,\mathrm{GL}(n,\mathbb{C}))],
\]

where \(\mathbf{FlatVect}_n\) denotes the category whose objects are
flat vector bundles of rank \(n\) and its morphisms are gauge
transformations. The category \(\mathbf{FlatVect}_n\) is clearly a
groupoid and it is also called the \emph{de Rham groupoid}. The set of
equivalence classes of this category is called the \emph{de Rham moduli
set} \(\mathcal{M}_{\text{de Rham}}\).

We can now state and prove the main theorem of this post:

\textbf{Theorem}. The functor we just gave factors through the inclusion
of local systems in principal bundles

\[
[Z^1(X,\underline{\mathrm{GL}(n,\mathbb{C})}), C^0(X,\underline{\mathrm{GL}(n,\mathbb{C}))}] \hookrightarrow [Z^1(X,\mathrm{GL}(n,\mathbb{C})), C^0(X,\mathrm{GL}(n,\mathbb{C}))],
\]

and the resulting functor

\[
\mathbf{FlatVect}_n \longrightarrow [Z^1(X,\underline{\mathrm{GL}(n,\mathbb{C})}), C^0(X,\underline{\mathrm{GL}(n,\mathbb{C})})],
\]

is an equivalence of categories. In particular, we get a bijection
\(\mathcal{M}_{\text{de Rham}} \cong \mathcal{M}_{\text{Betti}}\)
between the de Rham and the Betti moduli sets.

Since this is all a bit messy, let us state precisely what we are going
to prove:

\begin{enumerate}
\def\labelenumi{\arabic{enumi}.}
\item
  Provided any vector bundle \(E\), with a trivialization
  \((\mathfrak{U},\varphi)\) and a flat connection \(D\) on \(E\), we
  have to find locally constant functions
  \(h_{UV}:U\cap V \rightarrow \mathrm{GL}(n,\mathbb{C})\), for
  \(U,V \in \mathfrak{U}\) and \(U\cap V \neq \varnothing\) and
  functions \(f_U: U \rightarrow \mathrm{GL}(n,\mathbb{C})\) for
  \(U\in \mathfrak{U}\) such that \[
  h_{UV}=f_U g_{UV} f_V^{-1},
  \] where \(g_{UV}\) are the transition functions of \(E\) in the
  trivialization \((\mathfrak{U},\varphi)\). Moreover, we have to prove
  that \(g_{UV}\) and \(g_{UV}'\) are related by some \(0\)-cochain
  \(f \in C^0(\mathfrak{U},\mathrm{GL}(n,\mathbb{C}))\) if and only if
  the corresponding \(h_{UV}\) and \(h_{UV}'\) are related by some
  \(0\)-cochain
  \(\tilde{f} \in C^0(\mathfrak{U},\underline{\mathrm{GL}(n,\mathbb{C})})\).
  This accounts for the factorization and for being fully faithful.
\item
  Given any pair \((\mathfrak{U},h)\), with
  \(h\in Z^1(\mathfrak{U},\underline{\mathrm{GL}(n,\mathbb{C})})\), we
  have to construct a vector bundle \(E\) over \(X\) and a flat
  connection \(D\) on \(E\) such that the transition functions of \(E\)
  are given precisely by \(h\). This would show that the resulting
  functor is essentially surjective.
\end{enumerate}

Let us start by proving 1 by finding the mentioned locally constant
functions \(h_{UV}\). Suppose that we could find, for every
\(U\in \mathfrak{U}\), a frame

\begin{center}\rule{0.5\linewidth}{\linethickness}\end{center}

\href{https://guillegran.github.io/blog/indice.html}{Take me to the blog
index}

\href{https://guillegran.github.io}{Take me home}

\end{document}
